\documentclass[a4paper,12pt]{report}

\usepackage{alltt, fancyvrb, url}
\usepackage{graphicx}
\usepackage[utf8]{inputenc}
\usepackage{float}
\usepackage{xcolor}
\usepackage{hyperref}

\usepackage[italian]{babel}

\usepackage[italian]{cleveref}
\title{Relazione di Progetto \\ Web Server e Sito Web}
\author{Giovanni Maria Rava}
\date{\today}
\begin{document}
\maketitle
\tableofcontents
\chapter{Introduzione}
Questa relazione documenta lo sviluppo di un semplice Web Server scritto in linguaggio Python (versione 3.12), in grado di gestire 
richieste HTTP e servire contenuti statici in formato HTML e CSS. Il progetto ha finalità didattiche ed è stato realizzato nell’ambito
del corso di Programmazione di Reti (codice: 70226), durante l’Anno Accademico 2024/2025.

Durante lo sviluppo del progetto sono stati utilizzati diversi ambienti di lavoro e strumenti di sviluppo, elencati di seguito:
\begin{itemize}
    \item \textbf{Spyder 6}, per il debugging interattivo del codice Python;
    \item \textbf{Visual Studio Code}, per l’editing del codice e la gestione dei file HTML/CSS;
    \item \textbf{GitHub}, per il versionamento e l’archiviazione del progetto.
\end{itemize}
\section{Obbiettivo}
Progettare un semplice server HTTP in Python (usando socket) e servire un sito web statico con HTML/CSS.
\section{Requisiti minimi}
\begin{itemize}
    \item Il server deve rispondere su localhost:8080
    \item Deve servire almeno tre pagine HTML statiche
    \item Gestione di richieste GET e risposta con codice 200
    \item Implementare risposta 404 per file inesistenti 
\end{itemize}
\section{Estensioni opzionali}
\begin{itemize}
    \item Gestione dei MIME types (.html, .css, .jpg)
    \item Logging delle richieste
    \item Aggiunta di animazioni o layout responsive
\end{itemize}
\end{document}